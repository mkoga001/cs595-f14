\section{Problem 2}
\label{part2}
\subsection*{Question}
\begingroup
\begin{verbatim}

2.  Using these 100 files, create a single GraphViz "dot" file of
the resulting graph.  Learn about dot at:

Examples:
http://www.graphviz.org/content/unix
http://www.graphviz.org/Gallery/directed/unix.gv.txt

Manual:
http://www.graphviz.org/Documentation/dotguide.pdf

Reference:
http://www.graphviz.org/content/dot-language
http://www.graphviz.org/Documentation.php

Note: you'll have to put explicit labels on the graph, see:
https://gephi.org/users/supported-graph-formats/graphviz-dot-format/

(note: actually, I'll allow any of the formats listed here:

https://gephi.org/users/supported-graph-formats/

but "dot" is probably the simplest.)

\end{verbatim}
\newpage
\subsection{Answer}
\begin{enumerate}
\item To generate a DOT file I wrote a small Python program which reads all the files that are created in the first question.
\item I have generated a log file in the question 1 which contains a md5 hash value for the respective URI and I am gonna make use of that file now to get the URI for respective file name
\item I read all the files which are generated in the question 1 and map those links with the respective parent URI.
\item Write all the links in a DOT file(linksgraph.dot) along with labels for each link as required.
\end{enumerate}
\subsection{Code Listing}
\lstinputlisting[language=python, frame=single, breaklines=true,caption={Python Program for generating a DOT file}, label=lst:q1script, captionpos=b, numbers=left, showspaces=false, showstringspaces=false, basicstyle=\footnotesize]{q2/get_dot.py}