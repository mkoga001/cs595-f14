\section{Problem 1}
\label{part1}
\subsection*{Question}
\begingroup
\begin{verbatim}


1.  From your list of 1000 links, choose 100 and extract all of the
links from those 100 pages to other pages.  We're looking for user 
navigable links, that is in the form of: 

<A href="foo">bar</a>

We're not looking for embedded images, scripts, <link> elements,
etc.  You'll probably want to use BeautifulSoup for this.

For each URI, create a text file of all of the outbound links from
that page to other URIs (use any syntax that is easy for you).  For
example:

site: 
http://www.cs.odu.edu/~mln/    
links:
http://www.cs.odu.edu/
http://www.odu.edu/
http://www.cs.odu.edu/~mln/research/
http://www.cs.odu.edu/~mln/pubs/
http://ws-dl.blogspot.com/
http://ws-dl.blogspot.com/2013/09/2013-09-09-ms-thesis-http-mailbox.html
etc.

Upload these 100 files to github (they don't have to be in your report).

\end{verbatim}
\newpage
\subsection{Answer}

\begin{enumerate}
\item To get the 100 links, I did not pick them randomly from the 1000 links which we got from the Assignment 2. I just checked for the first 100 links 
\item I gave the unique links file as a command line argument and then read the lines only till I get 100 URIs which are working. 
\item I converted the URI to md5 hash so that I can save the files with the hash instead of URI which might have special characters 
\item By using beautiful soup I got the content which is there in anchor tag  and which are only href. 
\item I ignore the line which I got from the previous step if it does not start with http:// or https://.  
\item I save all the URIs that are collected by the above process in a file which is named with md5 hash value which we get by converting the parent URI to md5. 
\item All these 100 files are saved in a folder which has a name as the creation time of that folder. 
\end{enumerate}
\subsection{Code Listing}
\lstinputlisting[language=python, frame=single, caption={Python Program for printing out links from 100 pages}, label=lst:q1script, captionpos=b, numbers=left, showspaces=false, showstringspaces=false, basicstyle=\footnotesize]{q1/linkExract.py}