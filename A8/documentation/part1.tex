\section{Problem 1}
\label{part1}
\subsection*{Question}
\begingroup
\begin{verbatim}
The goal of this project is to use the basic recommendation principles
we have learned for user-collected data. You will modify the code
given to you which performs movie recommendations from the MovieLense
data sets.

The MovieLense data sets were collected by the GroupLens Research
Project at the University of Minnesota during the seven-month period
from September 19th, 1997 through April 22nd, 1998. It is available
for download from http://www.grouplens.org/node/73

There are three files which we will use:

1.  u.data: 100,000 ratings by 943 users on 1,682 movies. Each
user has rated at least 20 movies. Users and items are numbered
consecutively from 1. The data is randomly ordered. This is a tab
separated list of 

user id | item id | rating | timestamp

The time stamps are unix seconds since 1/1/1970 UTC.

Example:

196	242	3	881250949
186	302	3 	891717742
22	377	1 	878887116
244	51	2 	880606923
166	346	1 	886397596
298	474	4 	884182806
115	265	2	881171488

2.  u.item: Information about the 1,682 movies. This is a tab
separated list of

movie id | movie title | release date | video release date | IMDb URL | unknown | Action | Adventure
 | Animation |Children's | Comedy | Crime | Documentary | Drama | Fantasy | Film-Noir | Horror
 | Musical | Mystery | Romance | Sci-Fi | Thriller | War | Western |

The last 19 fields are the genres, a 1 indicates the movie is of
that genre, a 0 indicates it is not; movies can be in several genres
at once. The movie ids are the ones used in the u.data data set.

Example:

161|Top Gun (1986)|01-Jan-1986||http://us.imdb.com/M/title-exact?Top%20Gun%20(1986)
|0|1|0|0|0|0|0|0|0|0|0|0|0|0|1|0|0|0|0 
162|On Golden Pond (1981)|01-Jan-1981||http://us.imdb.com/M/title-exact?On%20Golden%20Pond%20(1981)
|0|0|0|0|0|0|0|0|1|0|0|0|0|0|0|0|0|0|0 
163|Return of the Pink Panther, The (1974)|01-Jan-1974||
http://us.imdb.com/M/title-exact?Return%20of%20the%20Pink%20Panther,%20The%20(1974)
|0|0|0|0|0|1|0|0|0|0|0|0|0|0|0|0|0|0|0

3.  u.user: Demographic information about the users. This is a tab
separated list of:

user id | age | gender | occupation | zip code

The user ids are the ones used in the u.data data set.

Example:

1|24|M|technician|85711 
2|53|F|other|94043 
3|23|M|writer|32067 
4|24|M|technician|43537 
5|33|F|other|15213

The code for reading from the u.data and u.item files and creating
recommendations is described in the book Programming Collective
Intelligence (check email for more details). You are to modify
recommendations.py to answer the following questions. Each question your
program answers correctly will award you 1 point.


What 5 movies have the highest average ratings? Show the movies
and their ratings sorted by their average ratings.
\end{verbatim}
\subsection{Solution}

\begin{enumerate}
\item I tried using the ``recommendations.py'' file which I downloaded from \url{https://github.com/arthur-e/Programming-Collective-Intelligence/blob/master/chapter2/recommendations.py}.
\item But I realized that this program can be used only when we need correlation or similarity or distance analysis. 
\item So I decided to write a piece of code which reads the respective file among the given 3 files to generate the output I want. 
\item Listing \ref{lst:q1} contains the source code for calculating the highest averages for movies. It runs like : 
\begin{lstlisting}[frame=single]
./highestRatings.py 11 
\end{lstlisting}
\item The argument is the number of the movies that are to be displayed. The result for the question asked is as follows 

\begin{lstlisting}[frame=single]
Entertaining Angels: The Dorothy Day Story (1996) 5.00
Star Kid (1997) 5.00
Great Day in Harlem, A (1994) 5.00
They Made Me a Criminal (1939) 5.00
Someone Else's America (1995) 5.00
Saint of Fort Washington, The (1993) 5.00
Aiqing wansui (1994) 5.00
Santa with Muscles (1996) 5.00
Prefontaine (1997) 5.00
Marlene Dietrich: Shadow and Light (1996)  5.00
Pather Panchali (1955) 4.62
\end{lstlisting}

\item From the above result we can see that top 10 movies have average rating of 5.0 and at the 11th movie the rating started going down to 4.62, so there are 10 movies with the average ratings as 5.
  
\end{enumerate}
\newpage
\lstinputlisting[language=python, frame=single,breaklines=true, caption={Python code for listing the movies with highest average ratings },  captionpos=b, numbers=left, showspaces=false,label=lst:q1, showstringspaces=false, basicstyle=\footnotesize]{questions/q1/highestRating.py}
\newpage